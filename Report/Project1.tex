\documentclass[letterpaper,11pt]{article}

\usepackage[open,openlevel=1]{bookmark}
\usepackage{amsmath}
\usepackage{graphicx}
\usepackage[ruled,vlined]{algorithm2e}
\usepackage{mcode}
\usepackage{subcaption}

% The percent sign indicates comments and does not effect what is written to the pdf

% If you would like to include figures use

% For inline equations, enclose with dollar signs ex. $\frac{1}{2}$

% For equation blocks use \begin{align} ... \end{align} for numbered equations and \begin{align*} \end{align*} for non numbered equations

% To include a figure, you can use the following command, or use it as a template
% To use this command type "\Figure{filename.png}{figure caption}{figure label}

\newcommand{\Figure}[3]{
%\Figure{File}{Caption}{Reference}
\begin{figure}[h]
\begin{center}
\includegraphics[width=3.5in]{#1}
\caption{#2}
\label{fig:#3}
\end{center}
\end{figure}
}

%%%%%%%%%%%%%%%%%%%%%%%%%%%%%%%%%%%%%%%%%%%%%%%%%%%%%%
%%%%%%%%%%%%%%%%% End of Preamble %%%%%%%%%%%%%%%%%%%%
%%%%%%%%%%%%%%%%%%%%%%%%%%%%%%%%%%%%%%%%%%%%%%%%%%%%%%

%%%%%%%%%%%%%%%%%%%%%%%%%%%%%%%%%%%%%%%%%%%%%%%%%%%%%%
%%%%%%%%%%% Document content start here %%%%%%%%%%%%%%
%%%%%%%%%%%%%%%%%%%%%%%%%%%%%%%%%%%%%%%%%%%%%%%%%%%%%%

\begin{document}

\title{Project 1: Fractal Geometry \\ 
		\large MAT128B Winter 2020}
\author{Caitlin Brown, Nikos Trembois, and Shuai Zhi}
\date{February 14, 2020}
\maketitle
\tableofcontents
\newpage

\section{Introduction}
In this project, numerical analysis, with the help of computers, is used to understand and demonstrate fractals and their characteristics. The fractals will be generated using orbits of complex numbers. Orbits are the process of applying a function to the output of the same function over and over, like a recursive function. It is not hard to imagine that certain initial values will produce diverging results while other will converge, or at least remain bounded by some value. The filled Julia set is all points whose orbit, using a polynomial function, remains bounded. The boundary of the filled set is called the Julia set.

\section{Introduction to Fractals}
The orbit of complex values whose real and imaginary part were within [-1, 1] were calculated for the function $\phi(z) = z^2$. In figure \ref{fig:unitDisk}, it is evident that the function $\phi = z^2$ maps the unit disk. The filled Julia set orbit of all points $z = a + ib$, where $a = [-1,1]$ and $b = [-1,1]$ under $\phi(z) = z^2$ creates the unit disk. Likewise, the Julia set under the same conditions would create the unit circle. 

\Figure{../Figures/UnitDisk.png}{Orbit of z under $\phi=z^2$}{fig:unitDisk}

By adding a constant to the function, fractals occur. 

\begin{figure}
\centering
	\begin{subfigure}[b]{0.49\textwidth}
		\includegraphics[width=\textwidth]{../Figures/FilledJulia1.png}
		\caption{Filled Julia Set of $z = 0.36 + 0.1i$}
		\label{fig:FJ+.36+.1i}
	\end{subfigure}
	\begin{subfigure}[b]{0.49\textwidth}
		\includegraphics[width=\textwidth]{../Figures/FilledJulia2.png}
		\caption{Filled Julia Set of $z = -0.123 + 0.745i$}
		\label{fig:FJ+.123+.745i}
	\end{subfigure}
	
	\vskip\baselineskip
	
	\begin{subfigure}[b]{0.49\textwidth}
		\includegraphics[width=\textwidth]{../Figures/FilledJulia3.png}
		\caption{Filled Julia Set of $z = - 0.749$}
		\label{fig:-.749}
	\end{subfigure}
	\begin{subfigure}[b]{0.49\textwidth}
		\includegraphics[width=\textwidth]{../Figures/FilledJulia4.png}
		\caption{Filled Julia Set of $z = -1.25$}
		\label{fig:FJ-1.25}
	\end{subfigure}
	\caption{Filled Julia Sets with 100 points in real and imaginary axis}
	\label{fig:FJwithC}
\end{figure}

\section{Julia Set}
Figure \ref{fig:FJ+.36+.1i}

\section{Fractal Dimensions}

\section{Julia Set Connectivity}

\section{Divergent Orbits}

\section{Newton's Method in Complex Plane}
Root finding can be a surprisingly difficult task. The linear case is trivial and roots of second order polynomials are solved with the quadratic equation. However, as the order increases the analytic equations to solve for the roots become more intricate and no known analytic equation exists for polynomials of orders higher than 6. Not to mention, this is just for real functions! Finding the roots of complex functions is even more difficult. Fortunately iterative methods, with the help of plots, simplify the process; although, some accuracy will be lost. The plots in figure \ref{fig:NI} are not just interesting, but also insightful. When plotting the Newton fractals and coloring the points with the number of iterations required to converge, the location of the roots become evident. For Newton's method, and any iterative method I can think of, the closer an initial guess is to a root, the less iterations are required. So the roots are found where the plots indicate the least amount of iterations are, in this case dark blue. Looking at figure \ref{fig:NI3} it appears the roots around $(1,0i), (-0.5,0.86i), (0.5,-0.86i)$. This agrees with the known values of $(1,0i), (-0.5, \frac{\sqrt{3}}{2}i), (-0.5, -\frac{\sqrt{3}}{2}i)$. The process extends to all subfigures in figure \ref{fig:NI} and to any complex function whose roots are desired.

\def\fwidth{0.49\textwidth}
\begin{figure}
\centering
	\begin{subfigure}[b]{\fwidth}
		\includegraphics[width=\textwidth]{../Figures/Newton1.png}
		\caption{Iterations to roots of $z^2 - 1$}
		\label{fig:NI2}
	\end{subfigure}
	\begin{subfigure}[b]{\fwidth}
		\includegraphics[width=\textwidth]{../Figures/Newton2.png}
		\caption{Iterations to roots of $z^3 - 1$}
		\label{fig:NI3}
	\end{subfigure}
	
	\vskip\baselineskip
	
	\begin{subfigure}[b]{\fwidth}
		\includegraphics[width=\textwidth]{../Figures/Newton3.png}
		\caption{Iterations to roots of $z^4 - 1$}
		\label{fig:NI4}
	\end{subfigure}
	\begin{subfigure}[b]{\fwidth}
		\includegraphics[width=\textwidth]{../Figures/Newton4.png}
		\caption{Iterations to roots of $z^5 - 1$}
		\label{fig:NI5}
	\end{subfigure}
	\caption{Roots of complex functions of form $z^n - 1$}
	\label{fig:NI}
\end{figure}

\section{The Mandelbrot Set}

\section{Conclusion}

\section{Appendix}

\subsection{Code}
\lstinputlisting[breaklines=true]{../Code/Project1.m}

\end{document}
